\textbf{Exercise 1.1:} Let $P$ be a probability measure on $(\Omega, \mathcal{F})$
\begin{enumerate}[i)]
    \item \textbf{Monotonicity:} If $A \subset B$ then $P(B) - P(A) = P(B-A) \geq 0$.
    \begin{mdframed}
        Let $C = B \cap A^c$. Then $A \cup C = B$. Because $A$ and $C$ are disjoint, we have that $P(B) = P(A \cup C) = P(A) + P(C)$. Then $P(B) - P(A) = P(C) = P(B \setminus A)$ which is greater or equal to zero since $P(C) \geq 0$.\qed
    \end{mdframed}
    \item \textbf{Subadditivity:} For $\langle A_m \rangle \in \mathcal{F}$ and $A \subset \cup^\infty A_m$ it follows that $P(A) \leq \sum^\infty P(A_m)$
    \begin{mdframed}
        Since $\langle A_m \rangle \in \mathcal{F}$ and $\mathcal{F}$ is a sigma field, there exists $\langle B_m \rangle$ such that $B_i \cap B_j = \emptyset$ for $i \neq j$ and $\cup^\infty A_m = \cup^\infty B_m$ \cite[17--18]{royden2nd}. From $(i)$ we know that $A \subset \cup B_m \implies P(A) \leq P(\cup B_m)$. Then \[P(A) \leq P(\cup^\infty B_m) = \sum^\infty P(B_m) \underset{\ast}{\leq} \sum^\infty P(A_m)\] The $(\ast)$ inequality follows from how we define each element of $\langle B_m\rangle$:
        \begin{align*}
            B_i = A_i \setminus [A_1 \cup \cdots A_{i-1}] &\implies P(B_i) = P(A_i) - \sum^{i-1}P(A_j) \\
            &\implies P(B_i) \leq P(A_i), \; \forall i
        \end{align*}\qed
    \end{mdframed}

    \item \textbf{Continuity from below:} If $A_i \uparrow A$ then $P(A_i) \uparrow P(A)$
    \begin{mdframed}
        By supposition, we know that for every $i$ we have $A_i = \cup^i_{j = 1} A_j$ 
        \[\lim_{i \to \infty} P(A_i)= \lim_{i \to \infty} P(\cup^i A_j) = P(A)\]
        (I'm not sure if this is rigorous enough but ...)\qed
    \end{mdframed}
    \item \textbf{Continuity from above:} If $A_i \downarrow A$ then $P(A_i) \downarrow P(A)$
    \begin{mdframed}
        For every $i$ we have that $A_i = \cap_{j = 1}^i A_j$
        \[\lim_{i \to \infty} P(A_j)= \lim_{i \to \infty} P(\cap^i A_j ) = P(A)\] \qed
    \end{mdframed}
\end{enumerate} 
\textbf{Exercise 1.2:} 
\begin{enumerate}[i)]
    \item If $\mathcal{F}_i, \; i\in I$ are $\sigma$-fields then $\cap_{i \in I}\mathcal{F}_i$ is.
    \begin{mdframed}
        Suppose $\mathcal{F}_i$ are $\sigma$-fields. 
        \begin{itemize}
            \item Consider $A \in \cap_{i \in I}\mathcal{F}_i$. Then $\exists i$ such that $A \in \mathcal{F}_i$. So $A^c \in \mathcal{F}_i$ because $\mathcal{F}_i$ is a $\sigma$-field $\sigma$-field. Then $A^c \in \cap_{i \in I}\mathcal{F}_i$. 
            \item Consider $A_j \in \cap_{i \in I}\mathcal{F}_i$ a countable sequence of sets. Then $\forall j, \forall i, \; A_j \in \mathcal{F}_i$. Then $\cup A_j \in F_i, \; \forall i$ because $\mathcal{F}_i$ is a $\sigma$-field. Then $\cup A_j \in \cap_{i \in I}\mathcal{F}_i$ 
        \end{itemize}\qed
        \end{mdframed}
    \item Use the result in (i) to show if we are given a set $\Omega$ and a collection $\mathcal{A}$ of subsets of $\Omega$ then there is a smallest $\sigma$-field containing $\mathcal{A}$.
    \begin{mdframed}
        Let $\mathcal{A}$ be a collection of subsets of $\Omega$. let $\mathcal{F}_\mathcal{A}$ be the set of sigma fields that contain $\mathcal{A}$. Define $\mathcal{F} = \cap_{\mathcal{A}}\mathcal{F}_\mathcal{A}$. From $(i)$ we know that $\mathcal{F}$ is a sigma field. By definition $\mathcal{F}$ is the smallest sigma field containing $\mathcal{A}$ since for sigma field $\mathcal{C}$ such that $\mathcal{A} \subset \mathcal{C}$, we have $\mathcal{F}=\cap_{\mathcal{A}}\mathcal{F}_\mathcal{A} \subset \mathcal{C}$\qed
    \end{mdframed}
\end{enumerate}

\noindent With $(\R, \mathcal{F}, P)$ and $\mathcal{B}$ the borel sets, define a random variable as a real valued function such that $X: \Omega \to \R$ is $\mathcal{F}$ measurable for every borel set \[X^{-1}(B)\in \mathcal{F}, \quad B \in \mathcal{B}\]
Then $X$ induces a probability measure on $\R$ called its distribution \[\mu(A) = P(X \in A) = P(X^{-1}(A)), \quad A \in \mathcal{B}\]
The \textbf{distribution function} is defined as \[F(x) = P(X \leq x)\]
When the distribution function $F(x) = P(X \leq x)$ has the form \[F(x) = \int_{-\infty}^x f(y)dy\] we say that $X$ has \textbf{density function} $f$.

\vspace{0.3cm}
\noindent\textbf{Exercise 1.5:} A $\sigma$-field $\mathcal{F}$ is said to be \textbf{countably generated} if there is a countable collection $\mathcal{C} \subset \mathcal{F}$ so that $\sigma(\mathcal{C})$. Show that $\mathcal{R}^d$ is countably generated.
\begin{mdframed}
    $\mathcal{R}^d$ are the Borel subsets of $\R^n$. First let's look at $\mathcal{R}$. We'll show that $\mathcal{G} = \{[q, \infty): q \in \Q\}$ generates  $\mathcal{R}$. Consider an arbitrary open interval $(a, b)\subset \R, \; a < b$. See that 
    \begin{equation}\label{open_int_ch1}
        [b, \infty)^c \cap \bigcup_{n}^\infty[a + 1/n, \infty) = (-\infty, b) \cap (a, \infty) = (a,b) 
    \end{equation}
    Remembering that every open set of real numbers is the countable union of disjoint open intervals \cite[42]{royden2nd}, we observe that using \ref{open_int_ch1} as a way to generate open intervals, we can also generate any open set. Therefore $\mathcal{R} \subset \sigma(\mathcal{G})$. To see that $\sigma (\mathcal{B}) \subset \mathcal{R}$, observe that any interval $[q, \infty)$ can be generated by unions, intersections, and complements of open sets (very easy to show). Therefore $\mathcal{R} = \sigma(\mathcal{G})$\\
    Since $\mathcal{G}$ is countable, $\mathcal{G} \times \cdots \times \mathcal{G}$ is countable and $\sigma(\mathcal{G} \times \cdots \times \mathcal{G}) = \mathcal{R} \times \cdots \times \mathcal{R} = \mathcal{R}^d$. \qed

\end{mdframed}
\textbf{Exercise 1.6:} Suppose $X$ and $Y$ are random variables on $(\Omega, \mathcal{F}, P)$ and let $A \in \mathcal{F}$. Show that if we let $Z(\omega) = X(\omega)$ for $\omega \in A$ and $Z(\omega) = Y\left( \omega \right)$ for $\omega \in A^c$, then $Z$ is a random variable.
\begin{mdframed}
    We want to show that $Z^{-1}(B) \in \mathcal{F}$ given $B \in \mathcal{R}$ (borel set). By supposition we have $Z^{-1}(B)\cap A = X^{-1}(B)\cap A$ and $Z^{-1}(B)\cap A^c = Y^{-1}(B)\cap A^c$. Therefore, 
    \begin{align*}
        Z^{-1}(B) &= (Z^{-1}(B) \cap A) \cup (Z^{-1}(B) \cap A^c) \\
        &= (X^{-1}(B) \cap A) \cup (Y^{-1}(B) \cap A^c) \overset{*}{\in} \mathcal{F}
    \end{align*}
    To show $(\ast)$, observe that since $X$ and $Y$ are random variables, we know that $X^{-1}(B), Y^{-1}(B)\in \mathcal{F}$. Then since $A, A^c \in \mathcal{F}$ it's clear that both $X^{-1}(B) \cap A$ and $Y^{-1}(B) \cap A^c \in \mathcal{F}$. So their intersection must also be an element of $\mathcal{F}$.\qed
\end{mdframed}
\textbf{Exercise 1.8:} Show that a distribution function has at most countably many discontinuities.
\begin{mdframed}
    Let $D = \{x \in \R: F(x-) \neq F(x+)\}$. Since $F$ is increasing, $x \in D \implies F(x-) < F(x+) \implies F(x+) - F(x-) > 0$. However, it's also clear from the fact that $F$ is increasing and $F(-\infty) = 0, F(\infty) = 1$ that $\sum_{x \in D} F(x+)-F(x-) \leq 1 < \infty$. Therefore $D$ must be countable infinite since the sum over the terms is finite \cite[11]{folland2nd}.\qed
\end{mdframed}
\begin{mdframed}
    \textit{Alternate solution partially inspired from a clever answer on stack exchange}\\
    Let $D$ be the set of points of discontinuity. Since $F(x-) < F(x+)$ for all $x \in D$, choose a rational $r_x$ such that $F(x-) < r_x < F(x+)$. Because $F$ is increasing, the intervals $(F(x-), F(x+))$ are mutually disjoint. So $x \to r_x$ is an injective function from $D \to \Q$. Therefore $D$ must be countably infinite. \qed
\end{mdframed}
\textbf{Exercise 1.9:} Show that if $F(x) = P(X \leq x)$ is continuous then $Y = F_X(X)$ has a uniform distribution on $(0,1)$. That is, if $y \in [0,1], \; P(Y \leq y) = y$
\begin{mdframed}
    Suppose $F_X(x) = P(X \leq x)$ is continuous. Let $Y = F_X(X)$. Then for $y \in [0,1]$
    \begin{align*}
        F_Y(y) &= P(Y \leq y) \\
        &= P(F_X(X) \leq y) \\
        &= P(X \leq F_X^{-1}(y)) &&\text{(using $F^{-1}$ on page 6)} \\
        &= F_X(F_X^{-1}(y))  = y
    \end{align*} \qed
\end{mdframed}
\textbf{Exercise 1.10:} Suppose that $X$ has density $f, \; P(\alpha \leq X \leq \beta) = 1$ and $g$ is a function that is increasing and differentiable on $(\alpha, \beta)$. Then $g(X)$ has density $f(g^{-1}(x))/g'(g^{-1}(x))$ for $x \in (g(\alpha), g(\beta))$ and $0$ otherwise. When $g(x) = ax + b$ with $a \geq 0$ the answer is $f((y-b)/a)/a$.
\begin{mdframed}
    Let $F_{g(X)}(x)$ be the distribution function for $g(x)$. Then the density function is $h = \frac{d}{dx}F_{g(X)}(x)$. Then  
    \begin{align*}
        F_{g(X)}(x) &= P(g(X) \leq x) \\
        &= P(X \leq g^{-1}(x)) && \text{(this exists because $g$ is increasing)} \\
        &= F_X(g^{-1}(x)) \\
    \end{align*}
    So by definition 
    \begin{align*}
        h = F_{g(X)}(x) &= \frac{d}{dx}F_X(g^{-1}(x)) \\
        &=f(g^{-1}(x)) \left(\frac{d}{dx}g^{-1}(x) \right)\\
        &=f(g^{-1}(x))\frac{1}{g^\prime(g^{-1}(x))} = \frac{f(g^{-1}(x))}{g^\prime(g^{-1}(x))} 
    \end{align*}
    Also notice that 
    \[P(\alpha \leq X \leq \beta)=1 \implies P(g(\alpha) \leq g(X) \leq g(\beta)) = 1 \implies x \in (g(\alpha), g(\beta))\]
\end{mdframed}
\textbf{Exercise 1.11:} Suppose $X$ has a normal distribution. Use the previous exercise to compute the density of $\exp(X)$. This is called the \textbf{lognormal distribution}.
\begin{mdframed}
    I'm assuming $X$ has a standard normal distribution ($\mu = 0, \, \sigma=1$). Then using the previous problem we have that $g(X) = \exp(X)$ has density
    \begin{align*}
        \frac{f(g^{-1} (x))}{g^\prime(g^{-1}(x))} &= \frac{f(\ln(x))}{\exp(\ln(x))}\\
        &= \frac{f(\ln(x))}{x} \\
        &= \frac{1}{x\sqrt{2\pi}}\exp\left(\frac{-\ln(x)^2}{2}\right), \quad \text{for $x \in (\exp(-\infty), \exp(\infty)) = (0, \infty)$}
    \end{align*}
\end{mdframed}
\textbf{Exercise 1.12:} 
\begin{enumerate}[i)]
    \item Suppose $X$ has density function $f$. Compute the distribution function of $X^2$ and then differentiate to find its density function.
    \item Work out the answer when $X$ has a standard normal distribution to find the density of the \textbf{chi-square distribution}.
\end{enumerate}
\begin{mdframed}
    
\end{mdframed}




