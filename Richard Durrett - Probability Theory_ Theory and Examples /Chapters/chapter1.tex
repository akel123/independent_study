\textbf{Exercise 1.1:} Let $P$ be a probability measure on $(\Omega, \mathcal{F})$
\begin{enumerate}[i)]
    \item \textbf{Monotonicity:} If $A \subset B$ then $P(B) - P(A) = P(B-A) \geq 0$.
    \begin{mdframed}
        Let $C = B \cap A^c$. Then $A \cup C = B$. Because $A$ and $C$ are disjoint, we have that $P(B) = P(A \cup C) = P(A) + P(C)$. Then $P(B) - P(A) = P(C) = P(B \setminus A)$ which is greater or equal to zero since $P(C) \geq 0$.
    \end{mdframed}
    \item \textbf{Subadditivity:} For $\langle A_m \rangle \in \mathcal{F}$ and $A \subset \cup^\infty A_m$ it follows that $P(A) \leq \sum^\infty P(A_m)$
    \begin{mdframed}
        Since $\langle A_m \rangle \in \mathcal{F}$ and $\mathcal{F}$ is a sigma field, there exists $\langle B_m \rangle$ such that $B_i \cap B_j = \emptyset$ for $i \neq j$ and $\cup^\infty A_m = \cup^\infty B_m$ \cite[17--18]{royden2nd}. From $(i)$ we know that $A \subset \cup B_m \implies P(A) \leq P(\cup B_m)$. Then \[P(A) \leq P(\cup^\infty B_m) = \sum^\infty P(B_m) \underset{\ast}{\leq} \sum^\infty P(A_m)\] The $(\ast)$ inequality follows from how we define each element of $\langle B_m\rangle$:
        \begin{align*}
            B_i = A_i \setminus [A_1 \cup \cdots A_{i-1}] &\implies P(B_i) = P(A_i) - \sum^{i-1}P(A_j) \\
            &\implies P(B_i) \leq P(A_i), \; \forall i
        \end{align*}
    \end{mdframed}

    \item \textbf{Continuity from below:} If $A_i \uparrow A$ then $P(A_i) \uparrow P(A)$
    \begin{mdframed}
        By supposition, we know that for every $i$ we have $A_i = \cup^i_{j = 1} A_j$ 
        \[\lim_{i \to \infty} P(A_i)= \lim_{i \to \infty} P(\cup^i A_j) = P(A)\]
        (I'm not sure if this is rigorous enough but ...)
    \end{mdframed}
    \item \textbf{Continuity from above:} If $A_i \downarrow A$ then $P(A_i) \downarrow P(A)$
    \begin{mdframed}
        For every $i$ we have that $A_i = \cap_{j = 1}^i A_j$
        \[\lim_{i \to \infty} P(A_j)= \lim_{i \to \infty} P(\cap^i A_j ) = P(A)\] 
    \end{mdframed}
\end{enumerate} 
\pagebreak
\textbf{Exercise 1.2:} 
\begin{enumerate}[i)]
    \item If $\mathcal{F}_i, \; i\in I$ are $\sigma$-fields then $\cap_{i \in I}\mathcal{F}_i$ is.
    \begin{mdframed}
        Suppose $\mathcal{F}_i$ are $\sigma$-fields. 
        \begin{itemize}
            \item Consider $A \in \cap_{i \in I}\mathcal{F}_i$. Then $\exists i$ such that $A \in \mathcal{F}_i$. So $A^c \in \mathcal{F}_i$ because $\mathcal{F}_i$ is a $\sigma$-field $\sigma$-field. Then $A^c \in \cap_{i \in I}\mathcal{F}_i$. 
            \item Consider $A_j \in \cap_{i \in I}\mathcal{F}_i$ a countable sequence of sets. Then $\forall j, \forall i, \; A_j \in \mathcal{F}_i$. Then $\cup A_j \in F_i, \; \forall i$ because $\mathcal{F}_i$ is a $\sigma$-field. Then $\cup A_j \in \cap_{i \in I}\mathcal{F}_i$ 
        \end{itemize}
        \end{mdframed}
    \item Use the result in (i) to show if we are given a set $\Omega$ and a collection $\mathcal{A}$ of subsets of $\Omega$ then there is a smallest $\sigma$-field containing $\mathcal{A}$.
    \begin{mdframed}
        Let $\mathcal{A}$ be a collection of subsets of $\Omega$. let $\mathcal{F}_\mathcal{A}$ be the set of sigma fields that contain $\mathcal{A}$. Define $\mathcal{F} = \cap_{\mathcal{A}}\mathcal{F}_\mathcal{A}$. From $(i)$ we know that $\mathcal{F}$ is a sigma field. By definition $\mathcal{F}$ is the smallest sigma field containing $\mathcal{A}$ since for sigma field $\mathcal{C}$ such that $\mathcal{A} \subset \mathcal{C}$, we have $\mathcal{F}=\cap_{\mathcal{A}}\mathcal{F}_\mathcal{A} \subset \mathcal{C}$
    \end{mdframed}
\end{enumerate}

\noindent With $(\R, \mathcal{F}, P)$ and $\mathcal{B}$ the borel sets, define a random variable as a real valued function such that $X: \Omega \to \R$ is $\mathcal{F}$ measurable for every borel set \[X^{-1}(B)\in \mathcal{F}, \quad B \in \mathcal{B}\]
Then $X$ induces a probability measure on $\R$ called its distribution \[\mu(A) = P(X \in A) = P(X^{-1}(A)), \quad A \in \mathcal{B}\]
The \textbf{distribution function} is defined as \[F(x) = P(X \leq x)\]
When the distribution function $F(x) = P(X \leq x)$ has the form \[F(x) = \int_{-\infty}^x f(y)dy\] we say that $X$ has \textbf{density function} $f$.

\noindent\textbf{Exercise 1.5:} A $\sigma$-field $\mathcal{F}$ is said to be \textbf{countably generated} if there is a countable collection $\mathcal{C} \subset \mathcal{F}$ so that $\sigma(\mathcal{C})$. Show that $\mathcal{R}^d$ is countably generated.
\begin{mdframed}
    $\mathcal{R}^d$ are the Borel subsets of $\R^n$. First let's look at $\mathcal{R}$. We'll show that $\mathcal{G} = \{[q, \infty): q \in \Q\}$ generates  $\mathcal{R}$. Consider an arbitrary open interval $(a, b)\subset \R, \; a < b$. See that 
    \begin{equation}\label{open_int_ch1}
        [b, \infty)^c \cap \bigcup_{n}^\infty[a + 1/n, \infty) = (-\infty, b) \cap (a, \infty) = (a,b) 
    \end{equation}
    Remembering that every open set of real numbers is the countable union of disjoint open intervals \cite[42]{royden2nd}, we observe that using \ref{open_int_ch1} as a way to generate open intervals, we can also generate any open set. Therefore $\mathcal{R} \subset \sigma(\mathcal{G})$. To see that $\sigma (\mathcal{B}) \subset \mathcal{R}$, observe that any interval $[q, \infty)$ can be generated by unions, intersections, and complements of open sets (very easy to show). Therefore $\mathcal{R} = \sigma(\mathcal{G})$\\
    Since $\mathcal{G}$ is countable, $\mathcal{G} \times \cdots \times \mathcal{G}$ is countable and $\sigma(\mathcal{G} \times \cdots \times \mathcal{G}) = \mathcal{R} \times \cdots \times \mathcal{R} = \mathcal{R}^d$

\end{mdframed}




